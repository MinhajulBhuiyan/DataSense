\documentclass[12pt,a4paper]{article}
\usepackage[utf8]{inputenc}
\usepackage[T1]{fontenc}
\usepackage{graphicx}
\usepackage{geometry}
\usepackage{hyperref}
\usepackage{listings}
\usepackage[dvipsnames]{xcolor}
\usepackage{tikz}
\usetikzlibrary{calc}
\usepackage{anyfontsize}
\usepackage{sectsty}
\usepackage{booktabs}
\usepackage{float}
\usepackage{enumitem}
\usepackage{amsmath}
\usepackage{fancyhdr}
\usepackage{titlesec}
\usepackage[most]{tcolorbox}
\usepackage{colortbl}
\usepackage{array}
\usepackage{pifont}
\usetikzlibrary{calc, positioning}

% Table of contents styling
\usepackage{tocloft}
% Color ToC title and spacing
\renewcommand{\cfttoctitlefont}{\color{datasensegreen}\huge\bfseries}
\renewcommand{\cftaftertoctitle}{\vspace{0.5em}}
% Make dot leaders and page numbers green
\renewcommand{\cftdot}{\textcolor{datasensegreen}{.}}
\renewcommand{\cftsecpagefont}{\color{datasensegreen}}
\renewcommand{\cftsubsecpagefont}{\color{datasensegreen}}
\renewcommand{\cftsubsubsecpagefont}{\color{datasensegreen}}

% Define custom colors - UPDATED PRIMARY COLOR
\definecolor{datasensegreen}{HTML}{00764c}
\definecolor{headerbg}{HTML}{00764c}
\definecolor{lightgreen}{HTML}{e8f5e9}
\definecolor{lightgray}{HTML}{6e6f73}
\definecolor{sectionbg}{HTML}{fff8e1}
\definecolor{tablebg}{HTML}{f9f9f9}
\definecolor{darkblue}{HTML}{1565C0}  % Darker blue
% Keybox-specific minimal blue (do not modify primary colors)
\definecolor{keyboxframe}{HTML}{0B66C3}
\definecolor{keyboxbg}{HTML}{F5FBFF}

\geometry{left=0.75in, right=0.75in, top=0.9in, bottom=0.9in, headheight=28pt}

% Code listing settings
\lstset{
    basicstyle=\ttfamily\small,
    breaklines=true,
    frame=single,
    backgroundcolor=\color{gray!10},
    keywordstyle=\color{datasensegreen},
    commentstyle=\color{green!60!black},
    stringstyle=\color{red},
    showstringspaces=false,
    frameround=tttt,
    rulecolor=\color{datasensegreen!50}
}

% Enhanced table styling
\newcolumntype{L}[1]{>{\raggedright\arraybackslash}p{#1}}
\newcolumntype{C}[1]{>{\centering\arraybackslash}p{#1}}
\newcolumntype{R}[1]{>{\raggedleft\arraybackslash}p{#1}}
\arrayrulecolor{datasensegreen!50}

% Header and footer
\pagestyle{fancy}
\fancyhf{}
\fancyhead[L]{\textcolor{lightgray}{\textbf{DataSense}}}
\fancyhead[R]{\textcolor{lightgray}{\textit{Technical Report}}}
\fancyfoot[C]{\textcolor{gray}{\small\thepage}}
\renewcommand{\headrulewidth}{0.8pt}
\renewcommand{\footrulewidth}{0pt}

% Colored header rule
\makeatletter
\renewcommand{\headrule}{{\color{lightgray}\hrule height \headrulewidth width\headwidth}}
\makeatother

\hypersetup{
    colorlinks=true,
    linkcolor=datasensegreen,
    filecolor=magenta,
    urlcolor=datasensegreen,
}

% Add green horizontal rule under section titles
\titleformat{\section}
  {\Large\bfseries\color{datasensegreen}}
  {\thesection}
  {1em}
  {}
  [\vspace{-0.5ex}{\color{datasensegreen}\rule{\linewidth}{0.8pt}}]

% Custom colored box for key sections
\newtcolorbox{keybox}[1][]{
    colback=keyboxbg,
    colframe=keyboxframe,
    colbacktitle=keyboxframe,
    coltitle=white,
    fonttitle=\bfseries,
    title=#1,
    boxrule=0.6pt,
    arc=2mm
}

\newtcolorbox{infobox}[1][]{
  colback=lightgreen,
  colframe=datasensegreen,
  fonttitle=\bfseries,
  title=#1,
  boxrule=0.5pt,
  arc=2mm,
  left=5pt,
  right=5pt,
  top=5pt,
  bottom=5pt
}

% Checkmark and cross symbols
\newcommand{\cmark}{\textcolor{datasensegreen}{\ding{51}}}
\newcommand{\xmark}{\textcolor{red}{\ding{55}}}
\newcommand{\pmark}{\textcolor{orange}{\ding{108}}}

\title{\textbf{DataSense: Natural Language to SQL Query Interface}\\
\large{A Comprehensive Technical Report}}
\author{Minhajul Abedin Bhuiyan \and Mahmudul Islam Mahin}
\date{January 6, 2026}

\begin{document}

% Custom TikZ Cover Page
\pagestyle{empty}

\begin{tikzpicture}[overlay,remember picture]

% Background color
\fill[black!2] (current page.south west) rectangle (current page.north east);

% Decorative rectangles with gradients - UPDATED COLORS
\shade[
left color=datasensegreen, 
right color=datasensegreen!40,
transform canvas ={rotate around ={45:($(current page.north west)+(0,-6)$)}}] 
($(current page.north west)+(0,-6)$) rectangle ++(9,1.5);

\shade[
left color=lightgray,
right color=lightgray!50,
rounded corners=0.75cm,
transform canvas ={rotate around ={45:($(current page.north west)+(.5,-10)$)}}]
($(current page.north west)+(0.5,-10)$) rectangle ++(15,1.5);

\shade[
left color=lightgray,
rounded corners=0.3cm,
transform canvas ={rotate around ={45:($(current page.north west)+(.5,-10)$)}}] 
($(current page.north west)+(1.5,-9.55)$) rectangle ++(7,.6);

\shade[
left color=datasensegreen!70,
right color=datasensegreen!50,
rounded corners=0.4cm,
transform canvas ={rotate around ={45:($(current page.north)+(-1.5,-3)$)}}]
($(current page.north)+(-1.5,-3)$) rectangle ++(9,0.8);

\shade[
left color=datasensegreen!80,
right color=datasensegreen!60,
rounded corners=0.9cm,
transform canvas ={rotate around ={45:($(current page.north)+(-3,-8)$)}}] 
($(current page.north)+(-3,-8)$) rectangle ++(15,1.8);

\shade[
left color=datasensegreen,
right color=datasensegreen!70,
rounded corners=0.9cm,
transform canvas ={rotate around ={45:($(current page.north west)+(4,-15.5)$)}}]
($(current page.north west)+(4,-15.5)$) rectangle ++(30,1.8);

\shade[
left color=darkblue,
right color=datasensegreen,
rounded corners=0.75cm,
transform canvas ={rotate around ={45:($(current page.north west)+(13,-10)$)}}]
($(current page.north west)+(13,-10)$) rectangle ++(15,1.5);

\shade[
left color=lightgray,
rounded corners=0.3cm,
transform canvas ={rotate around ={45:($(current page.north west)+(18,-8)$)}}]
($(current page.north west)+(18,-8)$) rectangle ++(15,0.6);

\shade[
left color=lightgray,
rounded corners=0.4cm,
transform canvas ={rotate around ={45:($(current page.north west)+(19,-5.65)$)}}]
($(current page.north west)+(19,-5.65)$) rectangle ++(15,0.8);

\shade[
left color=datasensegreen!90,
right color=datasensegreen!70,
rounded corners=0.6cm,
transform canvas ={rotate around ={45:($(current page.north west)+(20,-9)$)}}] 
($(current page.north west)+(20,-9)$) rectangle ++(14,1.2);

% Year label
\draw[ultra thick,gray]
($(current page.center)+(5,2)$) -- ++(0,-3cm) 
node[
midway,
left=0.25cm,
text width=5cm,
align=right,
black!75
]
{
{\fontsize{21}{30} \selectfont \bf TECHNICAL \\[10pt] REPORT}
} 
node[
midway,
right=0.25cm,
text width=6cm,
align=left,
datasensegreen]
{
{\fontsize{45}{86.4} \selectfont DLTD}
};

% Title and authors
\node[align=center] at ($(current page.center)+(0,-6)$) 
{
{\fontsize{50}{60} \selectfont \bf {{DataSense}}} \\[0.5cm]
{\fontsize{28}{33.6} \selectfont {{Natural Language to SQL}}} \\[1cm]
{\fontsize{16}{19.2} \selectfont \textcolor{datasensegreen}{ \bf Minhajul Abedin Bhuiyan}}\\[3pt]
{\fontsize{16}{19.2} \selectfont \textcolor{datasensegreen}{ \bf Mahmudul Islam Mahin}}};
% Logo at bottom
\node[align=center] at ($(current page.south)+(0,3)$) 
{
\includegraphics[width=3cm]{data-limited.png}\\[0.1cm]
};
\end{tikzpicture}



\newpage

%\maketitle
\thispagestyle{empty}


\vspace*{\fill}
\begin{abstract}
DataSense is a modern full-stack Natural Language to SQL (NL2SQL) application that enables non-technical users to query complex databases using plain English. Designed for Data Limited, the system provides an intelligent interface for data exploration, business intelligence, and reporting. This report presents a comprehensive technical overview of the system architecture, implementation details, key features, and design decisions that make DataSense a production-ready enterprise solution.
\end{abstract}
\vspace*{\fill}


\newpage
\tableofcontents
\newpage

% Restore fancy page style for main content
\pagestyle{fancy}

\section{Project Overview}

\subsection{What is DataSense?}
DataSense is a Natural Language–to–SQL application that enables users to query and analyze database systems using conversational English rather than structured SQL commands. The system is designed to support business and non-technical users by reducing the need for specialized database knowledge. By leveraging modern language models and web technologies, DataSense simplifies interaction with complex databases and facilitates efficient data access and analysis.

\subsection{Why Build DataSense?}
\begin{keybox}[Project Objectives]
\begin{enumerate}
    \item \textbf{Democratize Data Access}: Enable business users without SQL expertise to independently retrieve and analyze insights from database systems.
    \item \textbf{Ensure Data Safety}: Enforce read-only database access combined with comprehensive query validation to prevent unauthorized or harmful operations.
    \item \textbf{Provide Intelligence}: Leverage artificial intelligence to generate accurate, reliable, and context-aware SQL queries from natural language input.
    \item \textbf{Deliver Insights}: Support data-driven decision-making through automated and intelligent data visualization.
    \item \textbf{Scale Efficiently}: Ensure efficient handling of large datasets by incorporating data preview, filtering, and export mechanisms.
\end{enumerate}
\end{keybox}

\section{System Architecture}

\subsection{High-Level Architecture}
DataSense follows a modern three-tier architecture consisting of:

\begin{enumerate}
    \item \textbf{Presentation Layer}: Next.js-based React frontend with responsive UI
    \item \textbf{Application Layer}: Python Flask RESTful API server
    \item \textbf{Data Layer}: MySQL relational database with 23 normalized tables
    \item \textbf{AI Layer}: Ollama-powered LLM inference engine
\end{enumerate}

\subsection{Technology Stack}

\subsubsection{Frontend Technologies}
\begin{infobox}
\begin{itemize}
    \item \textbf{Framework}: Next.js 16 (App Router)
    \item \textbf{UI Library}: React 19
    \item \textbf{Language}: TypeScript 5
    \item \textbf{Styling}: Tailwind CSS 3.4
    \item \textbf{Charts}: Recharts 3.3
    \item \textbf{Theme}: next-themes for dark/light mode
    \item \textbf{State Management}: React hooks with localStorage persistence
\end{itemize}
\end{infobox}

\subsubsection{Backend Technologies}
\begin{infobox}
\begin{itemize}
    \item \textbf{Framework}: Flask 3.0
    \item \textbf{Language}: Python 3.x
    \item \textbf{Database Driver}: PyMySQL 1.1
    \item \textbf{Excel Export}: openpyxl 3.1
    \item \textbf{CORS}: flask-cors 4.0
    \item \textbf{Environment}: python-dotenv 1.0
\end{itemize}
\end{infobox}

\subsubsection{AI/ML Technologies}
\begin{infobox}
\begin{itemize}
    \item \textbf{LLM Engine}: Ollama
    \item \textbf{Models}: Llama 3 8B, Qwen 2.5 Coder, SQLCoder 7B, DeepSeek Coder 6.7B
    \item \textbf{Fine-tuning}: Optional LoRA adapters
    \item \textbf{Training}: Transformers, PyTorch
\end{itemize}
\end{infobox}

\newpage
\section{Version Control \& GitHub}

\begin{keybox}[Repository Health Summary]
\begin{itemize}
    \item \textbf{Repository}: 
    \href{https://github.com/MinhajulBhuiyan/DataSense}{github.com/MinhajulBhuiyan/DataSense}
\end{itemize}

\vspace{0.5em}

\begin{minipage}{0.48\linewidth}
\begin{itemize}
    \item \textbf{Status}: Active development
    \item \textbf{Branch}: \texttt{main}
\end{itemize}
\end{minipage}
\hfill
\begin{minipage}{0.48\linewidth}
\begin{itemize}
    \item \textbf{Total commits}: 43
    \item \textbf{Contributors}: 2
\end{itemize}
\end{minipage}
\end{keybox}



\section{Database Architecture}

\subsection{Schema Overview}

\begin{table}[H]
\centering
\rowcolors{2}{tablebg}{white}
\begin{tabular}{|L{4.5cm}|L{7.5cm}|C{2.5cm}|}
\hline
\rowcolor{headerbg}
\textcolor{white}{\textbf{Category}} & \textcolor{white}{\textbf{Description}} & \textcolor{white}{\textbf{Status}} \\ \hline
Schema Design & 23 normalized tables covering business operations, financials, inventory, and logistics & Complete \\
Data Integrity & Referential constraints and immutable recording patterns ensure data consistency and audit trails & Implemented \\ \hline
Security Model & Read-only enforcement with query validation and access controls & Active \\ \hline
Performance & Preview/export pattern with server-side streaming for large datasets & Optimized \\ \hline
\end{tabular}
\caption{Database Architecture Overview}
\end{table}

\noindent\textbf{Database Progress:}\hspace{2cm}\textcolor{datasensegreen}{\rule{0.5\linewidth}{8pt}}\hspace{0.5cm}\textbf{100\%}

\subsection{Key Design Principles}

\begin{itemize}
    \item \textbf{Data Immutability}: Original transaction records remain unchanged; adjustments recorded separately in dedicated tables for complete audit trail and historical accuracy
    \item \textbf{Referential Integrity}: All relationships between tables enforced through constraints ensuring data consistency across the system
    \item \textbf{Status Flow Control}: Business processes follow defined state transitions (orders, invoices, returns, refunds) preventing invalid operations
    \item \textbf{Read-Only Safety}: Application layer enforces query-only access protecting data from unauthorized modifications
\end{itemize}



\section{Backend Architecture}

\subsection{Flask API Server}

The backend is built on Flask 3.0, providing a lightweight yet powerful RESTful API that orchestrates all system operations. The architecture emphasizes modularity, security, and maintainability.

\subsection{API Endpoints}

The system exposes six primary REST endpoints for complete functionality:

\begin{table}[H]
\centering
\rowcolors{2}{tablebg}{white}
\begin{tabular}{|L{3.5cm}|L{2cm}|L{6cm}|}
\hline
\rowcolor{headerbg}
\textcolor{white}{\textbf{Endpoint}} & \textcolor{white}{\textbf{Method}} & \textcolor{white}{\textbf{Purpose}} \\ \hline
/api/query & POST & Processes natural language queries, returns SQL and results \\ \hline
/api/export & POST & Exports full query results to XLSX using token \\ \hline
/api/models & GET & Lists available Ollama models \\ \hline
/api/health & GET & System health check and database connectivity \\ \hline
/api/test & POST & Model comparison testing endpoint \\ \hline
/api/schema & GET & Returns database schema information \\ \hline
\end{tabular}
\caption{Backend API Endpoints}
\end{table}

\subsection{Backend Components \& Status}

\begin{table}[H]
\centering
\rowcolors{2}{tablebg}{white}
\begin{tabular}{|L{5cm}|L{6.5cm}|C{2cm}|}
\hline
\rowcolor{headerbg}
\textcolor{white}{\textbf{Component}} & \textcolor{white}{\textbf{Description}} & \textcolor{white}{\textbf{Status}} \\ \hline
\multicolumn{3}{|c|}{\textbf{Completed Components}} \\ \hline
Flask API Server & REST API with CORS and error handling & \cmark \\ \hline
Database Connector & PyMySQL connection with pooling & \cmark \\ \hline
Query Executor & Safe SQL execution with limits & \cmark \\ \hline
Query Validator & Whitelist/blacklist SQL validation & \cmark \\ \hline
Business Context Loader & Markdown file parser and loader & \cmark \\ \hline
Export Token Manager & Secure token generation and validation & \cmark \\ \hline
Health Check Endpoint & System status monitoring & \cmark \\ \hline
Model List Endpoint & Ollama model enumeration & \cmark \\ \hline
Error Handler & Centralized exception handling & \cmark \\ \hline
Environment Config & .env based configuration & \cmark \\ \hline
\multicolumn{3}{|c|}{\textbf{In Progress Components}} \\ \hline
Query Store & Persistent query history storage & \pmark \\ \hline
Schema Endpoint & Dynamic schema discovery & \pmark \\ \hline
\multicolumn{3}{|c|}{\textbf{Pending Components}} \\ \hline
Authentication Service & JWT-based user authentication & \xmark \\ \hline
Rate Limiter & Request throttling per user/IP & \xmark \\ \hline
Cache Layer & Redis-based response caching & \xmark \\ \hline
Async Task Queue & Celery for background jobs & \xmark \\ \hline
API Documentation & Swagger/OpenAPI spec & \xmark \\ \hline
Logging Service & Structured logging with rotation & \xmark \\ \hline
Metrics Collector & Prometheus metrics export & \xmark \\ \hline
\end{tabular}
\caption{Backend Components Implementation Status}
\end{table}

\noindent\textbf{Progress:}\hspace{1cm}\textcolor{datasensegreen}{\rule{0.40\linewidth}{8pt}}\textcolor{orange}{\rule{0.08\linewidth}{8pt}}\textcolor{red}{\rule{0.28\linewidth}{8pt}}\hspace{0.5cm}\textbf{58\%}

\subsection{Security Measures}

The backend implements multiple security layers:

\begin{itemize}
    \item \textbf{Read-Only Database User}: Prevents destructive operations at database level
    \item \textbf{SQL Injection Prevention}: Multi-layer validation with whitelist and blacklist
    \item \textbf{Query Keyword Filtering}: Blocks DELETE, UPDATE, INSERT, DROP, ALTER, TRUNCATE
    \item \textbf{CORS Configuration}: Controlled cross-origin access
    \item \textbf{Environment Isolation}: Secrets stored in .env files
    \item \textbf{Token-Based Export}: Secure export system with time-limited tokens
\end{itemize}

\subsection{Error Handling}

Comprehensive error handling ensures graceful degradation:

\begin{enumerate}
    \item \textbf{Database Connection Errors}: Automatic retry with exponential backoff
    \item \textbf{LLM Timeout Handling}: Fallback responses for slow model responses
    \item \textbf{Validation Errors}: Clear user-facing error messages
    \item \textbf{SQL Execution Errors}: Safe error reporting without exposing schema
    \item \textbf{Export Failures}: Token invalidation and user notification
\end{enumerate}


\section{AI and Natural Language Processing}

\subsection{LLM Integration}

\begin{keybox}[Supported LLM Models via Ollama]
\begin{enumerate}
    \item \textbf{Llama 3 8B (Default)}
    \begin{itemize}
        \item General-purpose model
        \item Excellent instruction following
        \item Balanced performance and accuracy
    \end{itemize}
    
    \item \textbf{Qwen 2.5 Coder}
    \begin{itemize}
        \item Code-specialized model
        \item Strong SQL generation capabilities
        \item Better for complex queries
    \end{itemize}
    
    \item \textbf{SQLCoder 7B}
    \begin{itemize}
        \item SQL-specific fine-tuned model
        \item Optimized for database queries
        \item High accuracy for standard SQL patterns
    \end{itemize}
    
    \item \textbf{DeepSeek Coder 6.7B}
    \begin{itemize}
        \item Efficient code generation
        \item Good balance of speed and accuracy
    \end{itemize}
\end{enumerate}
\end{keybox}

\noindent\textbf{Model Integration Status:}\hspace{2cm}\textcolor{datasensegreen}{\rule{0.5\linewidth}{8pt}}\hspace{0.5cm}\textbf{100\%}

\subsection{Prompt Engineering}

\subsubsection{Prompt Structure}
The system constructs prompts with three main components:

\begin{enumerate}
    \item \textbf{Business Context}: Domain-specific rules and definitions
    \item \textbf{Database Schema}: Relevant table and column information
    \item \textbf{User Query}: The natural language question
\end{enumerate}

\subsubsection{Smart Schema Filtering}
To optimize token usage and improve accuracy, the system focuses intelligent schema filtering:
\begin{enumerate}
    \item Detects which tables are relevant to the user's question.
    \item Includes only those table definitions in the prompt.
    \item Reduces token usage and improves model accuracy without exposing the full schema.
\end{enumerate}

\subsection{Training Process}
\begin{enumerate}
    \item Curated dataset of 120 business queries.
    \item Automated script builds compact LoRA adapter.
    \item Enhances accuracy with fast deployment.
\end{enumerate}

\subsection{AI Components \& Pipeline Status}

\begin{table}[H]
\centering
\rowcolors{2}{tablebg}{white}
\begin{tabular}{|L{5cm}|L{6.5cm}|C{2cm}|}
\hline
\rowcolor{headerbg}
\textcolor{white}{\textbf{Component}} & \textcolor{white}{\textbf{Description}} & \textcolor{white}{\textbf{Status}} \\ \hline
\multicolumn{3}{|c|}{\textbf{Completed Components}} \\ \hline
LLM Integration & Multi-model support via Ollama & \cmark \\ \hline
Business Context Loader & Domain knowledge injection & \cmark \\ \hline
Query Validator & SQL safety and injection checks & \cmark \\ \hline
Input Validation & User query validation & \cmark \\ \hline
SQL Cleaner & Markdown/formatting removal & \cmark \\ \hline
Query Executor & Safe execution with limits & \cmark \\ \hline
\multicolumn{3}{|c|}{\textbf{In Progress Components}} \\ \hline
Prompt Builder & Dynamic LLM prompt construction & \pmark \\ \hline
Query Processor & NL to SQL conversion pipeline & \pmark \\ \hline
Result Formatter & Data response formatting & \pmark \\ \hline
Dataset Generator & Business query examples & \pmark \\ \hline
Smart Schema Filter & Context-aware table detection & \pmark \\ \hline
\multicolumn{3}{|c|}{\textbf{Pending Components}} \\ \hline
LoRA Training Script & Automated fine-tuning & \xmark \\ \hline
Model Caching & Response caching system & \xmark \\ \hline
Query Suggestion & Auto-complete predictions & \xmark \\ \hline
Voice Input & Speech-to-text integration & \xmark \\ \hline
Multi-turn Context & Conversation history tracking & \xmark \\ \hline
Query Explanation & Natural language SQL breakdown & \xmark \\ \hline
Anomaly Detection & Result pattern analysis & \xmark \\ \hline
\end{tabular}
\caption{AI Components Implementation Status}
\end{table}

\noindent\textbf{Progress:}\hspace{1cm}\textcolor{datasensegreen}{\rule{0.25\linewidth}{8pt}}\textcolor{orange}{\rule{0.21\linewidth}{8pt}}\textcolor{red}{\rule{0.29\linewidth}{8pt}}\hspace{0.5cm}\textbf{40\%}


\section{User Interface Design}
\subsection{Design Philosophy}
DataSense UI follows modern design principles:
\begin{itemize}
    \item \textbf{Simplicity}: Clean, uncluttered interface
    % \item \textbf{Responsiveness}: Mobile-first design
    % \item \textbf{Accessibility}: ARIA labels, keyboard shortcuts
    \item \textbf{Feedback}: Real-time status indicators
    \item \textbf{Consistency}: Unified color scheme and typography
\end{itemize}


\subsection{Key UI Components}

\subsubsection{Chat Interface}
\begin{itemize}
    %\item Message bubbles with role-based styling
    \item User messages: right-aligned, dark background
    \item Assistant messages: centered, light background
    \item SQL queries: syntax-highlighted code blocks
    \item Results: responsive data tables
\end{itemize}

\subsubsection{Input Area}
\begin{itemize}
    %\item Auto-resizing textarea (max 200px height)
    \item Send button (enabled when text present)
    \item Stop button (during query execution)
    %\item Keyboard shortcuts (Enter to send, Shift+Enter for newline)
\end{itemize}

\subsubsection{Sidebar}
\begin{itemize}
    \item Collapsible design 
    \item Conversation list with rename/delete actions
    \item New chat button
    \item Settings modal trigger
    \item Theme toggle
    \item Connection status indicator
\end{itemize}

\subsection{Data Visualization}

\subsubsection{Intelligent Chart Detection}
The system analyzes result data to suggest appropriate chart types:

    \begin{itemize}
        \item \textbf{Bar Chart}: Categorical + numeric columns
        \item \textbf{Horizontal Bar}: Many categories (better readability)
        \item \textbf{Pie Chart}: Few categories (2-8) with proportions
        \item \textbf{Line Chart}: Temporal + numeric (trend analysis)
        \item \textbf{Area Chart}: Temporal + numeric (volume emphasis)
        \item \textbf{Scatter Plot}: Two numeric columns (correlation)
    \end{itemize}
    

\subsection{UI Components \& Status}

\begin{table}[H]
\centering
\rowcolors{2}{tablebg}{white}
\begin{tabular}{|L{5cm}|L{6.5cm}|C{2cm}|}
\hline
\rowcolor{headerbg}
\textcolor{white}{\textbf{Component}} & \textcolor{white}{\textbf{Description}} & \textcolor{white}{\textbf{Status}} \\ \hline
\multicolumn{3}{|c|}{\textbf{Completed Components}} \\ \hline
Design Prompting Interface & Structured prompt input with validation & \cmark \\ \hline
Theme System & Light/dark mode with next-themes & \cmark \\ \hline
Color Scheme & Primary green, accent colors (blue, amber, red) & \cmark \\ \hline
Chat Interface & Message bubbles, SQL highlighting, result tables & \cmark \\ \hline
Input Area & Auto-resize textarea, send/stop controls, shortcuts & \cmark \\ \hline
Sidebar & Collapsible, conversations, settings, theme toggle & \cmark \\ \hline
Bar/Line/Pie Charts & Basic chart types with Recharts & \cmark \\ \hline
Area/Scatter Charts & Volume emphasis and correlation plots & \cmark \\ \hline
Chart Auto-suggestion & Intelligent chart type detection & \cmark \\ \hline
Interactive Tooltips & Hover-based data display & \cmark \\ \hline
\multicolumn{3}{|c|}{\textbf{In Progress Components}} \\ \hline
Responsive UI & Mobile-first Tailwind CSS refinements & \pmark \\ \hline
Advanced Charts & Heatmaps, treemaps planned & \pmark \\ \hline
\multicolumn{3}{|c|}{\textbf{Pending Components}} \\ \hline
Mobile App & PWA or React Native packaging & \xmark \\ \hline
Extended Accessibility & Full ARIA compliance, screen reader testing & \xmark \\ \hline
\end{tabular}
\caption{UI Components Implementation Status}
\end{table}

\noindent\textbf{Progress:}\hspace{1cm}\textcolor{datasensegreen}{\rule{0.54\linewidth}{8pt}}\textcolor{orange}{\rule{0.11\linewidth}{8pt}}\textcolor{red}{\rule{0.11\linewidth}{8pt}}\hspace{0.5cm}\textbf{72\%}


\newpage
\section{Performance Optimization}

\subsection{Performance Components \& Status}

\begin{table}[H]
\centering
\rowcolors{2}{tablebg}{white}
\begin{tabular}{|L{5cm}|L{6.5cm}|C{2cm}|}
\hline
\rowcolor{headerbg}
\textcolor{white}{\textbf{Component}} & \textcolor{white}{\textbf{Description}} & \textcolor{white}{\textbf{Status}} \\ \hline
\multicolumn{3}{|c|}{\textbf{Completed Components}} \\ \hline
Preview/Export Pattern & 50-row preview with full XLSX export & \cmark \\ \hline
Server-Side Cursor & Streaming for large dataset exports & \cmark \\ \hline
Connection Auto-Reconnect & Automatic DB reconnection on failure & \cmark \\ \hline
\multicolumn{3}{|c|}{\textbf{In Progress Components}} \\ \hline
Query Wrapping & Automatic LIMIT application & \pmark \\ \hline
\multicolumn{3}{|c|}{\textbf{Pending Components}} \\ \hline
Redis Caching & Cache repeated query results & \xmark \\ \hline
Virtual Scrolling & Render large tables efficiently & \xmark \\ \hline
Result Pagination & Navigate large result sets & \xmark \\ \hline
WebSocket Updates & Real-time query execution status & \xmark \\ \hline
Query Timeout Handling & Cancel long-running queries & \xmark \\ \hline
Memory Management & Limit result set memory usage & \xmark \\ \hline
Database Connection Pool & Reuse connections efficiently & \xmark \\ \hline
\end{tabular}
\caption{Performance Optimization Implementation Status}
\end{table}

\noindent\textbf{Progress:}\hspace{1.5cm}\textcolor{datasensegreen}{\rule{0.15\linewidth}{8pt}}\textcolor{orange}{\rule{0.05\linewidth}{8pt}}\textcolor{red}{\rule{0.50\linewidth}{8pt}}\hspace{0.5cm}\textbf{37\%}


\newpage
\section{Testing and Quality Assurance}

\subsection{Testing Components \& Status}

\begin{table}[H]
\centering
\rowcolors{2}{tablebg}{white}
\begin{tabular}{|L{5cm}|L{6.5cm}|C{2cm}|}
\hline
\rowcolor{headerbg}
\textcolor{white}{\textbf{Component}} & \textcolor{white}{\textbf{Description}} & \textcolor{white}{\textbf{Status}} \\ \hline
\multicolumn{3}{|c|}{\textbf{Completed Components}} \\ \hline
Model Comparison Tool & Multi-model accuracy and timing tests & \cmark \\ \hline
Connection Diagnostics & Database and Ollama connectivity tests & \cmark \\ \hline
SQL Injection Tests & Security validation for query inputs & \cmark \\ \hline
Manual QA Testing & Core workflow validation and bug fixes & \cmark \\ \hline
\multicolumn{3}{|c|}{\textbf{In Progress Components}} \\ \hline
Query Validation Tests & Edge cases for SQL generation & \pmark \\ \hline
\multicolumn{3}{|c|}{\textbf{Pending Components}} \\ \hline
Unit Tests & Backend functions and React components & \xmark \\ \hline
API Integration Tests & REST endpoint validation & \xmark \\ \hline
End-to-End Tests & Automated user workflow testing & \xmark \\ \hline
Export Functionality Tests & XLSX generation and download & \xmark \\ \hline
Chart Rendering Tests & Visualization accuracy validation & \xmark \\ \hline
Load Testing & Concurrent user simulation & \xmark \\ \hline
Error Handling Tests & Edge case and failure scenarios & \xmark \\ \hline
Responsive Design Tests & Multi-device layout validation & \xmark \\ \hline
\end{tabular}
\caption{Testing \& QA Implementation Status}
\end{table}

\noindent\textbf{Progress:}\hspace{1.5cm}\textcolor{datasensegreen}{\rule{0.22\linewidth}{8pt}}\textcolor{orange}{\rule{0.05\linewidth}{8pt}}\textcolor{red}{\rule{0.45\linewidth}{8pt}}\hspace{0.5cm}\textbf{30\%}



\section{Future Enhancements}

\subsection{Planned Features}
\begin{enumerate}
    \item \textbf{Query History}: Global search across all conversations
    \item \textbf{Query Suggestions}: Auto-complete based on schema
    \item \textbf{Scheduled Reports}: Periodic query execution and email delivery
    \item \textbf{User Authentication}: Role-based access control
    \item \textbf{Query Sharing}: Share queries with team members
    \item \textbf{Advanced Visualizations}: More chart types (heatmaps, treemaps)
    \item \textbf{Natural Language Follow-ups}: Context-aware query refinement
    \item \textbf{Export Formats}: PDF, CSV in addition to XLSX
    \item \textbf{Data Caching}: Redis-based query result caching
    \item \textbf{Multi-Database Support}: PostgreSQL, SQL Server
\end{enumerate}

\subsection{Performance Improvements}
\begin{itemize}
    \item Query result pagination
    \item Virtual scrolling for large tables
    \item WebSocket for real-time updates
    \item CDN integration for frontend assets
    \item Database read replicas for scaling
\end{itemize}

\subsection{AI Enhancements}
\begin{itemize}
    \item Multi-turn conversations with context retention
    \item Query explanation in natural language
    \item Anomaly detection in results
    \item Predictive analytics suggestions
    \item Voice input support
\end{itemize}

\newpage
\section{Challenges and Obstacles}

Throughout the development of DataSense, the team encountered various technical, operational, and learning-related challenges. This section categorizes the key obstacles faced during implementation, ranked by severity.

\subsection{Challenge Classification}

\subsubsection{CRITICAL Challenges}

\begin{tcolorbox}[colback=red!10, colframe=red!70, title=\textbf{Critical Severity - Project-Blocking Issues}, boxrule=0.8pt, arc=2mm]
\begin{itemize}
    
    \item \textbf{Database Connection Failures}: Intermittent network connectivity issues between application and MySQL server necessitated connection pooling and automatic reconnection strategies
    

    \item \textbf{Hardware Barrier - GPU Unavailability}: Lack of GPU resources on deployment server prevented local model training, necessitating reliance on pre-trained models and external Ollama service

    \item \textbf{Time Wasted on Template Solutions}: Backend template-based handlers and Parser reduced flexibility; switched to an AI-first custom implementation.

   \item \textbf{Version Control Issues}: While working directly on the deployment server prevented smooth committing and pushing changes to GitHub.
    
    \item \textbf{Server Downtime}: Unpredictable outages in the company’s internal server infrastructure disrupted development and testing cycles.

    \item \textbf{Insufficient Monitoring \& Observability}: Missing centralized logs, metrics, and alerts, causing slow incident detection

    \item \textbf{Single-Point-of-Failure}: Core services lacked redundancy, risking total outage

\end{itemize}
\end{tcolorbox}

\subsubsection{MAJOR Challenges}

\begin{tcolorbox}[colback=orange!10, colframe=orange!70, title=\textbf{Major Severity - Significant Impact Issues}, boxrule=0.7pt, arc=2mm]
\begin{itemize}
   
    \item \textbf{New Technology Exploration}: Limited prior experience with Ollama, LoRA fine-tuning, and NL2SQL systems required extensive research, experimentation, and learning
    
    \item \textbf{Frequent Requirement Changes}: Evolving business needs and feature requests required flexible architecture and iterative development approach

    \item \textbf{LLM API Rate Limiting}: Model inference timeouts and rate limits during peak usage required implementation of queuing systems and timeout handling    
    
    \item \textbf{LLM Accuracy and Consistency}: Inconsistent SQL generation quality across different models required extensive prompt engineering and model comparison testing
    
    \item \textbf{Large Dataset Performance}: Initial implementations struggled with result sets exceeding 10,000 rows, requiring preview/export pattern and server-side streaming
    
    \item \textbf{Schema Complexity Management}: Complex 23-table schema with intricate relationships made it difficult for LLMs to generate accurate queries without smart filtering
    
    \item \textbf{State Management Complexity}: Managing conversation history, theme preferences, and connection status across components required careful React hook design
\end{itemize}
\end{tcolorbox}

\subsubsection{MINOR Challenges}

\begin{tcolorbox}[colback=yellow!10, colframe=yellow!70, colbacktitle=yellow!70, coltitle=black, fonttitle=\bfseries, title=\textbf{Minor Severity - Manageable Issues}, boxrule=0.6pt, arc=2mm]
\begin{itemize}
    \item \textbf{Initial Documentation Gaps}: Lack of comprehensive business context documentation at project start required iterative refinement of markdown files
      
    \item \textbf{Framework Learning Curve}: Team unfamiliarity with Python libraries and backend tooling required dedicated learning time

    \item \textbf{Git Merge Conflicts}: Concurrent development on overlapping features caused frequent merge conflicts, resolved through better branch management and communication
    
    \item \textbf{Environment Configuration}: Managing different .env configurations across development, testing, and deployment environments caused occasional debugging delays
    
    \item \textbf{Color Scheme Iterations}: Multiple revisions to achieve the final green color palette and dark mode compatibility
    
    \item \textbf{Export Format Limitations}: Initial CSV export implementation had encoding issues, requiring switch to XLSX format
    
    \item \textbf{TypeScript Type Errors}: Strict type checking in TypeScript occasionally slowed development but improved code quality
\end{itemize}
\end{tcolorbox}

\subsection{Mitigation Strategies}

To address these challenges, the team implemented several key strategies:

\begin{itemize}
    
    \item \textbf{Iterative Development}: Frequent feedback loops and a flexible architecture to accommodate changing requirements
    
    \item \textbf{Extensive Testing}: Model comparison framework, security validation, and cross-browser testing to ensure reliability
    
    \item \textbf{Performance Optimization}: Preview/export pattern, server-side streaming, and smart schema filtering to handle large datasets
    
    \item \textbf{Team Communication}: Regular standups, code reviews, and knowledge sharing sessions to minimize conflicts and improve collaboration
    
\end{itemize}

\newpage
\section{Overall Progress Summary}

This section provides a comprehensive overview of the DataSense project's implementation status, tracking completion across all major components and layers.

\subsection{Progress by Layer}

The following table summarizes completion status across the system architecture:

\begin{table}[H]
\centering
\rowcolors{2}{tablebg}{white}
\begin{tabular}{|L{5cm}|L{6.5cm}|C{2cm}|}
\hline
\rowcolor{headerbg}
\textcolor{white}{\textbf{Layer/Component}} & \textcolor{white}{\textbf{Description}} & \textcolor{white}{\textbf{Status}} \\ \hline
\multicolumn{3}{|c|}{\textbf{Completed Layers}} \\ \hline
Database Architecture & 23 normalized tables, referential integrity, read-only safety & \cmark \\ \hline
\multicolumn{3}{|c|}{\textbf{In Progress Layers}} \\ \hline
Backend API (58\%) & 10 components done, 2 in-progress, 7 pending & \pmark \\ \hline
User Interface (72\%) & 10 components done, 2 in-progress, 2 pending & \pmark \\ \hline
AI \& NLP (40\%) & 6 components done, 5 in-progress, 7 pending & \pmark \\ \hline
Performance Optimization (15\%) & 2 components done, 1 in-progress, 11 pending & \pmark \\ \hline
Testing \& QA (15\%) & 2 components done, 1 in-progress, 11 pending & \pmark \\ \hline
\multicolumn{3}{|c|}{\textbf{Pending Layers}} \\ \hline
Authentication System & JWT, OAuth2, RBAC, role-based access & \xmark \\ \hline
Production Deployment & Docker containers, Kubernetes orchestration & \xmark \\ \hline
Advanced Caching & Redis integration, query result caching & \xmark \\ \hline
API Documentation & Swagger/OpenAPI specification & \xmark \\ \hline
CI/CD Pipeline & Automated testing and deployment & \xmark \\ \hline
Monitoring \& Logging & Prometheus metrics, structured logging & \xmark \\ \hline
\end{tabular}
\caption{Implementation Progress by System Layer}
\end{table}

\noindent\textbf{Overall System Progress:}\hspace{0.5cm}\textcolor{datasensegreen}{\rule{0.20\linewidth}{8pt}}\textcolor{orange}{\rule{0.10\linewidth}{8pt}}\textcolor{red}{\rule{0.30\linewidth}{8pt}}\hspace{0.5cm}\textbf{35\%}



\subsection{Completion Breakdown}

\subsubsection{Completed}

\begin{infobox}[Fully Implemented Features]
\begin{itemize}
    \item \textbf{Natural language to SQL conversion} using Ollama LLMs
    \item \textbf{Multi-model support} (Llama 3, Qwen, SQLCoder, DeepSeek)
    \item \textbf{Database schema design} (23 normalised tables with referential integrity)
    \item \textbf{Read-only enforcement} for database safety
    \item \textbf{Business context integration} (5 markdown files loaded dynamically)
    \item \textbf{Six REST API endpoints} (query, export, models, health, test, schema)
    \item \textbf{SQL validation system} (whitelist/blacklist, injection prevention)
    \item \textbf{Preview and export pattern} (50-row preview, full XLSX export)
    \item \textbf{Token-based export system} with secure generation and validation
    \item \textbf{Server-side cursor streaming} for large datasets
    \item \textbf{Intelligent chart generation} (bar, line, pie, scatter, area)
    \item \textbf{Interactive data visualization} with auto-suggestion and tooltips
    \item \textbf{Conversation management} (up to 20 conversations, rename/delete)
    \item \textbf{Theme system} (light/dark mode with next-themes)
    \item \textbf{All React components} (8 components: ChatInput, Message, Sidebar, etc.)
    \item \textbf{Custom hooks} (useConnectionStatus, useConversations, useTheme)
    \item \textbf{Connection status monitoring} with auto-reconnect
    \item \textbf{Model comparison testing tool} for accuracy and performance validation
    \item \textbf{SQL injection security tests} passed
    \item \textbf{Centralized error handling} across all endpoints
    \item \textbf{Environment-based configuration} (.env for secrets)
    \item \textbf{Health monitoring endpoint} for system status
    \item \textbf{Training dataset} (120 curated business queries)
    \item \textbf{Technical documentation} (comprehensive LaTeX report)
    \item \textbf{GitHub repository} with version control
\end{itemize}
\end{infobox}

\subsubsection{In Progress (Partially Completed)}

\begin{tcolorbox}[colback=yellow!10, colframe=orange!60, title=\textbf{Active Development}, boxrule=0.6pt, arc=2mm]
\begin{itemize}
    \item \textbf{Backend components} (Query Store, Schema Endpoint)
    \item \textbf{AI pipeline components} (Prompt Builder, Query Processor, Result Formatter)
    \item \textbf{Smart schema filtering} (Context-aware table detection)
    \item \textbf{Dataset generation} (Automated training query generation)
    \item \textbf{Responsive UI design} (Mobile-first Tailwind CSS refinements)
    \item \textbf{Advanced chart types} (Heatmaps, treemaps implementation)
    \item \textbf{Query wrapping optimization} (Schema-aware LIMIT application)
    \item \textbf{Cross-browser testing} (Chrome, Firefox, Safari, Edge validation)
\end{itemize}
\end{tcolorbox}

\subsubsection{Pending}

\begin{tcolorbox}[colback=red!5, colframe=red!40, title=\textbf{Future Roadmap}, boxrule=0.6pt, arc=2mm]
\begin{itemize}
    \item \textbf{User authentication system} (JWT, OAuth2, RBAC)
    \item \textbf{Backend services} (Rate Limiter, Logging Service, Metrics Collector)
    \item \textbf{Redis caching layer} (Query result and response caching)
    \item \textbf{Async task queue} (Celery for background jobs)
    \item \textbf{API documentation} (Swagger/OpenAPI specification)
    \item \textbf{LoRA training automation} (Fine-tuning pipeline)
    \item \textbf{Advanced NLP features} (Multi-turn context, Query explanation, Voice input)
    \item \textbf{Model enhancements} (Model caching, Query suggestions, Anomaly detection)
    \item \textbf{Performance optimization} (Virtual scrolling, Pagination, CDN, Connection pooling)
    \item \textbf{Extended testing} (Unit tests, Integration tests, E2E tests, Performance tests)
    \item \textbf{CI/CD pipeline} (Automated testing and deployment)
    \item \textbf{Multi-database support} (PostgreSQL, SQL Server adapters)
    \item \textbf{Query sharing and collaboration} (Share links, team workspaces)
    \item \textbf{Mobile app} (PWA or React Native packaging)
    \item \textbf{Admin dashboard} (User management, query analytics)
    \item \textbf{Production deployment} (Docker containers, Kubernetes orchestration)
    \item \textbf{Advanced security} (Audit logging, Data masking, Real-time collaboration)
\end{itemize}
\end{tcolorbox}

% \subsection{Timeline and Milestones}

\newpage
\section{Conclusion}

DataSense successfully demonstrates how modern AI technologies can democratize data access in enterprise environments. By combining state-of-the-art LLMs with robust engineering practices, the system provides a safe, efficient, and user-friendly interface for database querying.

\subsection{Key Achievements}
\begin{itemize}
    \item \textbf{User-Friendly Interface}: Non-technical users can query databases
    \item \textbf{Safety First}: Comprehensive validation prevents data corruption
    \item \textbf{High Performance}: Preview and export pattern handles large datasets
    \item \textbf{Intelligent Visualization}: Automatic chart generation
    \item \textbf{Extensible Architecture}: Easy to add new models and features
    \item \textbf{Production-Ready}: Complete error handling and monitoring
\end{itemize}

\subsection{Technical Excellence}
The project showcases best practices in:
\begin{itemize}
    \item Full-stack development with modern frameworks
    \item RESTful API design
    \item Database schema design and normalization
    \item AI/ML integration
    \item Security and validation
    \item Performance optimization
    \item User experience design
\end{itemize}

\subsection{Business Value}
DataSense provides significant business value:
\begin{itemize}
    \item Reduces dependency on technical staff for data queries
    \item Enables faster decision-making with instant data access
    \item Improves data literacy across the organization
    \item Reduces errors from manual SQL writing
    \item Scales easily to handle growing data volumes
    \item Provides audit trail for all queries
\end{itemize}

\end{document}
